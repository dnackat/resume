%-------------------------
% Resume in Latex
% Author : Sourabh Bajaj (https://github.com/sb2nov/resume/blob/master/sourabh_bajaj_resume.tex)
% Edited by: Dileep Nackathaya
% License : MIT
%------------------------

\documentclass[letterpaper,11pt]{article}

\usepackage{latexsym}
\usepackage[empty]{fullpage}
\usepackage{titlesec}
\usepackage{marvosym}
\usepackage[usenames,dvipsnames]{color}
\usepackage{verbatim}
\usepackage{enumitem}
\usepackage[hidelinks]{hyperref}
\usepackage{fancyhdr}
\usepackage[english]{babel}

\pagestyle{fancy}
\fancyhf{} % clear all header and footer fields
\fancyfoot{}
\renewcommand{\headrulewidth}{0pt}
\renewcommand{\footrulewidth}{0pt}

% Adjust margins
\addtolength{\oddsidemargin}{-0.5in}
\addtolength{\evensidemargin}{-0.5in}
\addtolength{\textwidth}{1in}
\addtolength{\topmargin}{-.5in}
\addtolength{\textheight}{1.0in}

\urlstyle{same}

\raggedbottom
\raggedright
\setlength{\tabcolsep}{0in}

% Sections formatting
\titleformat{\section}{
	\vspace{-4pt}\scshape\raggedright\large
}{}{0em}{}[\color{black}\titlerule \vspace{-5pt}]

%-------------------------
% Custom commands
\newcommand{\resumeItem}[2]{
	\item\small{
		\textbf{#1}{: #2 \vspace{-2pt}}
	}
}

\newcommand{\resumeSubheadingMod}[5]{
	\vspace{-1pt}\item
	\begin{tabular*}{0.97\textwidth}[t]{l@{\extracolsep{\fill}}r}
		\textbf{#1} & #2 \\
		\textit{\small#3} & \textit{\small #4} \\
		\textit{\small#5} \\
	\end{tabular*}\vspace{-5pt}
}

\newcommand{\resumeSubheading}[4]{
	\vspace{-1pt}\item
	\begin{tabular*}{0.97\textwidth}[t]{l@{\extracolsep{\fill}}r}
		\textbf{#1} & #2 \\
		\textit{\small#3} & \textit{\small #4} \\
	\end{tabular*}\vspace{-5pt}
}


\newcommand{\resumeSubItem}[2]{\resumeItem{#1}{#2}\vspace{-4pt}}

\renewcommand{\labelitemii}{$\circ$}

\newcommand{\resumeSubHeadingListStart}{\begin{itemize}[leftmargin=*]}
	\newcommand{\resumeSubHeadingListEnd}{\end{itemize}}
\newcommand{\resumeItemListStart}{\begin{itemize}}
	\newcommand{\resumeItemListEnd}{\end{itemize}\vspace{-5pt}}

%-------------------------------------------
%%%%%%  CV STARTS HERE  %%%%%%%%%%%%%%%%%%%%%%%%%%%%


\begin{document}
	
	%----------HEADING-----------------
	\begin{tabular*}{\textwidth}{l@{\extracolsep{\fill}}r}
		\textbf{\Large Dileep Nackathaya} & Email : \href{mailto:dileepbn@gmail.com}{dileepbn@gmail.com}\\
		\href{https://github.com/dnackat/}{https://github.com/dnackat/} & Mobile : +91-7899129478 \\
		\href{https://www.linkedin.com/in/dnackat/}{https://www.linkedin.com/in/dnackat/} \\
	\end{tabular*}
	
	
	%-----------EDUCATION-----------------
	\section{Education}
	\resumeSubHeadingListStart
	\resumeSubheadingMod
	{North Carolina State University}{Raleigh, NC, USA}
	{Master of Science in Mechanical Engineering;  GPA: 3.75/4.00}{Aug 2010 -- Dec 2012}
	{Specialization: Computational Fluid Dynamics (CFD)}{}
	\resumeSubheading
	{Visvesvaraya Technological University}{Belgaum, India}
	{Bachelor of Engineering in Mechanical Engineering;  Grade: First Class (74\%)}{Sep. 2006 -- July. 2010}
	\resumeSubHeadingListEnd
	
	%----------LEARNING--------------------
	\section{Continuous Learning}
	\resumeSubHeadingListStart
	\resumeSubheading
	{Statistics, Machine Learning, Data Science}{Udupi, India}
	{Self-learning (progress documented on LinkedIn and GitHub)}{Jan 2018 -- Present}
	\resumeItemListStart
	\resumeItem{\href{https://www.edx.org/micromasters/mitx-statistics-and-data-science/}{Statistics and Data Science MicroMasters (offered by MITx on edX)}}
	{Four graduate level credit-eligible courses with challenging assignments and projects in Probability, Statistics, Data Analysis, and Machine Learning along with a comprehensive Capstone exam \textit {\small (Skills: Python, R, PyTorch, NumPy, SciPy, Matplotlib, Scikit-learn).}}
	\resumeItem{\href{https://www.coursera.org/specializations/deep-learning/}{Deep Learning Specialization (offered by deeplearning.ai on Coursera)}}
	{Currently pursuing this program with five courses on fundamental Deep Learning algorithms. (\textit{\small Skills: Python, Tensorflow}).}
	\resumeItem{\href{https://www.coursera.org/learn/machine-learning/}{Machine Learning (taught by Prof. Andrew Ng on Coursera)}}
	{An introductory machine learning course with eight programming projects (\textit{\small Skills: MATLAB/GNU Octave}).}
	\resumeItem{Other courses}
	{\href{https://www.edx.org/course/the-analytics-edge-2/}{The Analytics Edge}, \href{https://www.edx.org/course/introduction-to-r-for-data-science-3/}{Intro to R for Data Science}, \href{https://www.coursera.org/learn/sql-for-data-science/}{SQL for Data Science},  \href{https://www.edx.org/course/using-python-for-research-2}{Using Python for Research},  \href{https://www.edx.org/course/cs50s-introduction-computer-science-harvardx-cs50x/}{CS50: Intro to Computer Science}, \href{https://www.edx.org/course/6-00-1x-introduction-to-computer-science-and-programming-using-python-3/}{Intro to Computation and Programming using Python}}
	\resumeItemListEnd
	\resumeSubHeadingListEnd

	%-----------EXPERIENCE-----------------
	\section{Experience}
	\resumeSubHeadingListStart
	
	\resumeSubheading
	{\href{https://www.johnzinkhamworthy.com/}{John Zink Hamworthy Combustion}}{Tulsa, OK, USA}
	{Computational Fluid Dynamics Engineer, R \& D Group}{Jun 2013 -- Aug 2017}
	\resumeItemListStart
	\resumeItem{Simulation and Analysis}
	{Created CFD models of industrial burners, flares, thermal oxidizers, and vapor recovery systems and analyzed simulation data. Prepared customer reports on findings of these analyses.}
	\resumeItem{Product Development}
	{Leveraged data from CFD simulations and analyses to provide insights on designing new products and improving existing ones.}
	\resumeItem{Troubleshooting}
	{Analyzed data from customer sites and ran simulations to troubleshoot on-site product issues.}
	\resumeItemListEnd
	\resumeSubHeadingListEnd
	
	%-----------PROJECTS-----------------
	\section{Projects}
	\resumeSubHeadingListStart
	\resumeSubItem{Digit Recognition}
	{Used multiclass SVM, softmax regression, and convolutional neural networks to recognize single and overlapping digits. Compared performance of these algorithms using different metrics. (\textit{\small Python, Scikit-learn, PyTorch})}
	\resumeSubItem{Automatic Review Analyzer}
	{Used Perceptron and Pegasos algorithms for sentiment analysis of Amazon reviews. Used cross-validation for hyperparameter tuning and did feature engineering to improve performance. (\textit{\small Python, Numpy})}
	\resumeSubItem{Netflix Movie Ratings}
	{Used the EM algorithm to generate Gaussian mixtures for collaborative filtering to predict movie ratings and compared it to k-Means clustering. Used Bayesian Information to pick clusters. (\textit{\small Python, Numpy})}
	\resumeSubItem{Reinforcement Learning}
	{Taught an agent to play a simple game using the parameterized Q-learning algorithm. Implemented a neural network to learn the parameters for maximal reward. (\textit{\small Python, Numpy, PyTorch, Matplotlib})}
	\resumeSubItem{Predicting Office Space Prices}
	{Implemented multivariate polynomial regression from scratch in Python for predictions including formatting the dataset, gradient descent algorithm, hyperparameter tuning, and visualization.}
	\resumeSubItem{Spam Detection}
	{Used kernelized SVM algorithm to build a spam classifier in GNU Octave including preprocessing email text and extracting features for training.}
	\resumeSubItem{Statistical Analysis using R}
	{Used data from Social Science studies to compute p-values, confidence sets, and test hypotheses. Set up multivariate linear models and visualized results with ggplot2.}
	\resumeSubHeadingListEnd
	
	%-----------SKILLS-----------------
	\section{Technical Skills}
	\resumeSubHeadingListStart
	\resumeSubItem{Programming and Scripting Languages}
	{Python, R, MATLAB/GNU Octave, SQL, C, Shell, Fortran}
	\resumeSubItem{Operating Systems}
	{GNU/Linux, Windows}
	\resumeSubItem{Version Control}
	{Git. Used a bit of SVN in the past.}
	\resumeSubItem{Libraries and Packages}
	{NumPy, Pandas, SciPy, Matplotlib, Scikit-learn, PyTorch, Tensorflow, NLTK, ggplot2}
	\resumeSubItem{High Performance Computing}
	{Used AWS and company/university clusters to do large parallel computations.}
	\resumeSubHeadingListEnd
	
\end{document}